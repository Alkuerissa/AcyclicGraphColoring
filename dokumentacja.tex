\documentclass{article}
\usepackage{polski}
\usepackage[utf8]{inputenc}

\title{\texttt{Problem acyklicznego kolorowania grafu}}
\author{Ahmed Abdelkarim, Aleksandra Hernik}
\begin{document}
\maketitle
\section{Definicja problemu}
Zadanie polega na pokolorowaniu grafu tak, aby każde dwa kolory indukowały graf acykliczny. 
\section{Algorytmy}
Przedstawione zostaną trzy przybliżone algorytmy.
\subsection{Algorytm zachłanny ze sprawdzeniami w każdym kroku}
\begin{enumerate}
\item Stwórz pusty graf $G'$ - będą do niego dodawane pokolorowane wierzchołki.
\item Dla każdego wierzchołka $v$:
\item Dodaj $v$ do $G'$.
\item Stwórz zbiór zakazanych kolorów, zawierający kolory wszystkich sąsiadów $v$ w $G'$.
\item Znajdź wszystkie cykle w $G'$, do których należy $v$. Jeśli któryś zawiera wierzchołki tylko w dwóch kolorach, dodaj te kolory do zbioru zakazanych.
\item Pokoloruj wierzchołek na pierwszy niezakazany kolor.
  
\end{enumerate}
\subsection{Algorytm zachłanny na wstępnie przygotowanym grafie}

\subsection{Poprawienie wyniku algorytmu zachłannego}

\subsection{Dopełnienie do grafu triangulowanego}

\end{document}
