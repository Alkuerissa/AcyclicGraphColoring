\documentclass{article}
\usepackage{polski}
\usepackage[utf8]{inputenc}
\usepackage{hyperref}

\title{Problem acyklicznego kolorowania grafu}
\author{Ahmed Abdelkarim, Aleksandra Hernik}
\begin{document}
\maketitle
\section{Definicja problemu}
Zadanie polega na pokolorowaniu grafu tak, aby każde dwa kolory indukowały graf acykliczny. 
\section{Algorytmy}
Przedstawione zostaną trzy przybliżone algorytmy.
\subsection{Algorytm zachłanny ze sprawdzeniami w każdym kroku}
\begin{enumerate}
\item Stwórz pusty graf $G'$ - będą do niego dodawane pokolorowane wierzchołki.
\item Dla każdego wierzchołka $v$:
\item Dodaj $v$ do $G'$.
\item Stwórz zbiór zakazanych kolorów, zawierający kolory wszystkich sąsiadów $v$ w $G'$.
\item Znajdź wszystkie cykle o parzystej długości w $G'$, do których należy $v$. Jeśli któryś zawiera wierzchołki tylko w dwóch kolorach, dodaj te kolory do zbioru zakazanych.
\item Pokoloruj wierzchołek na pierwszy niezakazany kolor.
  
\end{enumerate}
\subsection{Poprawienie wyniku algorytmu zachłannego}
\begin{enumerate}
\item Pokoloruj graf dowolną metodą.
\item Znajdź $Z$ - zbiór wszystkich cykli o parzystej długości.
\item Dla każdego $C \in Z$ sprawdź, czy jest pokolorowany na dokładnie dwa kolory.
\item Przekoloruj jeden wierzchołek z $C$ na inny kolor (jeśli to możliwe, kolor, który już został użyty w grafie), usuń z $Z$ wszystkie cykle, do których należał ten wierzchołek. W celu wybrania koloru należy przejść przez wszystkie cykle, do których należał ten wierzchołek (włącznie z tymi, które już zostały usunięte z $Z$) i jeśli dany cykl (nie wliczając tego wierzchołka) był pokolorowany na dokładnie dwa kolory, zakazać ich. Następnie wybrać pierwszy kolor, który nie był zakazany oraz nie jest kolorem któregoś z sąsiadów.
\end{enumerate}
Różne metody wyboru wierzchołka do przekolorowania mogą dać różne rezultaty. Metodą, która wydaje się dobrym przybliżeniem jest:
\begin{enumerate}
\item Wyznacz zbiór wierzchołków $V_p \subseteq C$, które można przekolorować nie zwiększając liczby użytych kolorów.
\item Jeśli $V_p = \O$, przyjmij $V_p = C$.
\item Wybierz wierzchołek, który jest częścią największej liczby cykli w $Z$. 
\end{enumerate}
\subsection{Dopełnienie do grafu triangulowanego}
Gebremedhin i inni udowodnili, że każde poprawne kolorowanie grafu triangulowanego, czyli takiego, że każdy cykl o długości większej niż 3 ma krawędź łączącą dwa wierzchołki z tego cyklu (ale niebędącą jego częścią), jest kolorowaniem acyklicznym. \cite{Gb} Ponadto, w takim grafie kolorowanie jest problemem rozwiązywalnym w czasie wielomianowym, bo to graf doskonały.
Problem dopełnienia grafu do grafu triangulowanego przy najmniejszej możliwej liczbie dodanych krawędzi jest problemem NP-zupełnym \cite{NPC}, o złożoności $O(1.9601^n)$ \cite{Cmp}, tak więc w celu zwiększenia szybkości algorytmu, który i tak jest algorytmem przybliżonym, można wykorzystać do dopełnienia algorytm aproksymacyjny. Algorytmem o złożoności wielomianowej, który zapewnia dodanie co najwyżej $8k^2$ krawędzi, gdzie $k$ to minimalna liczba krawędzi w dopełnieniu, jest algorytm opracowany przez Assafa Natanzona, Rona Shamira, i Rodeda Sharana \cite{Approx}. Inną możliwością jest zastosowanie jednego z wielu algorytmów, które znajdują dopełnienie grafu do grafu triangulowanego bez żadnej gwarancji co do liczby dodanych krawędzi, na przykład MCS-M \cite{MCS-M}. Oprócz wielu znanych algorytmów, do zastosowania w problemie kolorowania acyklicznego wydaje się mieć sens proste podejście:
\begin{enumerate}
\item Znajdź wszystkie cykle o długości większej niż 3.
\item Znajdź w każdym cyklu wierzchołek, który należy do największej liczby cykli.
\item Dodaj w każdym cyklu krawędzie między znalezionym wierzchołkiem a wszystkimi innymi wierzchołkami cyklu (jeśli ich jeszcze nie było).
\item Znajdź $Z$ - zbiór wszystkich cykli o długości większej niż 3.
\item Dla każdego cyklu $C \in Z$:
\item Znajdź wierzchołek $v$, który należy do największej liczby cykli w $Z$.
\item Dla każdego cyklu $C_v$, do którego należy $v$, dodaj krawędzie miedzy $v$ a pozostałymi wierzchołkami $C_v$ (o ile jeszcze ich nie było) i usuń $C_v$ z $Z$.
\end{enumerate}
\section{Uwagi}
\subsection{Znajdowanie cykli w grafie}
Znalezienie cykli w grafie jest częścią każdego z opisanych algorytmów. 
Warto zauważyć, że w opisywanym problemie wystarczy znalezienie cykli niezawierających krawędzi między dwoma wierzchołkami cyklu, która nie jest jego częścią, a nie wszystkich. Wszystkie inne cykle będą zawierały wyłącznie wierzchołki z co najmniej dwóch takich cykli. 
Wynika z tego, że jeśli takie cykle będą spełniały założenia zadania, to wszystkie inne też.
Można w tym celu wykorzystać implementację zaproponowaną w wypowiedzi \cite{Cycles}.



\begin{thebibliography}{20}
\bibitem{MCS-M}
Berry, A.; Blair, J. R. S.; Heggernes P.; Peyton B. W. (2004), \textit{Maximum cardinality search for computing minimal triangulations of graphs}.
Algorithmica, 39: 287–298

\bibitem{Cmp}
Fomin, Fedor V.; Kratsch, Dieter; Todinca, Ioan (2004), \textit{Exact (exponential) algorithms for treewidth and minimum fill-In}, Automata, Languages and Programming: 31st International Colloquium, ICALP 2004, Turku, Finland, July 12-16, 2004, Proceedings, Lecture Notes in Computer Science, 3142, Springer-Verlag, pp. 568–580

\bibitem{Gb}
Gebremedhin, Assefaw H.; Tarafdar, Arijit; Pothen, Alex; Walther, Andrea (2008), \textit{Efficient Computation of Sparse Hessians Using Coloring and Automatic Differentiation}, Informs Journal on Computing, 21: 209

\bibitem{Approx}
Natanzon, Assaf; Shamir, Ron; Sharan, Roded (2000), \textit{A polynomial approximation algorithm for the minimum fill-in problem}, SIAM Journal on Computing, 30 (4): 1067–1079

\bibitem{NPC}
Yannakakis, Mihalis (1981), \textit{Computing the minimum fill-in is NP-complete}, Society for Industrial and Applied Mathematics, 2 (1): 77–79

\bibitem{Cycles}
\url{http://stackoverflow.com/questions/4022662/find-all-chordless-cycles-in-an-undirected-graph?noredirect=1\&lq=1}

\end{thebibliography}
\end{document}
